\subsection*{Feedback fra Lars}
\textit{Dejligt. Jeg tror dog ikke, at “2 nestede for-loops" holder i det
generelle tilfælde — og det vil Christian eller Pawel (læs: sikkert Pawel)
muligvis slå ned på. Generelt har du dog en god struktur på dit oplæg, og
overgangen fra et punkt til det næste er meget organisk. Jeg vil dog, i
stedet for at bevise korrektheden af algoritmen ved en løkkeinvariant,
anbefale dig at bevise den optimale delstruktur for det rekursive udtryk.
Korrektheden af algoritmen tager på en måde udgangspunkt i dette, og
beviset for optimal delstruktur er et bevis, der i mine øjne er bedre
egnet til et mundtligt oplæg.}

\begin{itemize}
    \item Dynamic Programming
    \begin{itemize}
        \item Divide the problem into subproblem
        \item Solve the subproblems only once and save the solution
        \item Top-down memoization: recursion
        \item Bottom-up: Two nested \texttt{for}-loops
    \end{itemize}
    \item Longest common subsequence
    \begin{itemize}
        \item Finds a string that both the initial strings have in common
        \item Demonstration by figure
        \item Correctness - Optimal substructure
        \item Runtime $\Theta(mn)$
    \end{itemize}\iffalse
    \item Perspective - Greedy algorithms
    \begin{itemize}
        \item Optimises in a different way
    \end{itemize}\fi
\end{itemize}
