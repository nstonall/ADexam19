\subsection*{Amortized Analysis (Stack operations)}
Instead of finding the worst case time of a single operation, we wish to find
the worst case time of $n$ operations.

\subsubsection*{Aggregate analysis}
We determine an upper bound $T(n)$ on the total cost of a sequence of $n$
operation. We take the average cost as the amortized cost of each operation, so
that all operations have the same amortized cost. Even several types of
operations will have the same amortized cost.
\newline\newline\textit{Stack operations}\newline
For a sequence of $n$ \texttt{PUSH}, \texttt{POP} and \texttt{MULTIPOP}
operations on an empty stack, all operations have a total cost of $O(n)$, since
the number of calls within \texttt{MULTIPOP}, and the number of times
\texttt{POP} is called, cannot exceed the number of \texttt{PUSH operations},
which is at most $n$. The average cost of each operation is $O(n)/n=O(1)$.

\subsubsection*{Accounting method}
In the accounting method we charge early operations more than their cost. That
way we charge less in later opeartions. The amount we charge each opeartions, is
its amortized cost, and we save the extra charge as "prepaid credit". If we
undercharge early on, the credit becomes negative, and the total amortized cost
incurred at that time will not be an upper bound. We must therefore not
undercharge in the beginning.
\newline\newline\textit{Stack operations}\newline
We charge the \texttt{PUSH} operation for both the \texttt{PUSH} and
\texttt{POP} operations, and such the amortized cost of the \texttt{PUSH}
operation is $2$ and the amortized cost for both \texttt{POP} and
\texttt{MULTIPOP} is $0$. This means that the amortized cost of each operation
is in constant time and the total amortized cost is $O(n)$, which is also the
actual upper bound worst-case cost.

\subsubsection*{Potential method}
The potential method works in the same way as the accounting method, but
maintains the extra credit as "potential energy". This is associated with the
data structure as a whole, instead of individual operations. The potential
energy of a initial data structure is denoted as $\Phi(D_0)$. The potential
energy of a data structure after the $i$th operation is denoted as $\Phi(D_i)$.
The amortized cost $\hat{c_i}$ of the $i$th operation is calculated as such:
\begin{align*}
  \hat{c_i}=c_i+\Phi(D_i)-\Phi(D_{i-1})
\end{align*}
The total amortized cost of $n$ operations is:
\begin{align*}
  \sum^n_{i=1}\hat{c_i}&=\sum^n_{i=1}(c_i+\Phi(D_i)-\Phi(D_{i-1}))\\
  &=\sum^n_{i=1}(c_i)+\Phi(D_n)-\Phi(D_{0})
\end{align*}
If $\Phi(D_i)-\Phi(D_{i-1})$ yields a positive result, then the amortized cost
$\hat{c_i}$ represents an overcharge. If it is negative it represents an
undercharge.
\newline\newline\textit{Stack operations}\newline
The potential function $\Phi$ denotes the number of objects in the stack. It
can never be negative. \newline\newline
We say that $\Phi(D_i)\geq\Phi(D_0)$. We let $s$ denote the amount of objects
in the stack. If the potential difference is positive, then there are more
objects in the stack than before. \newline\newline
We let the $i$th operation be a \texttt{PUSH} operation on a stack of size $s$.
\begin{align*}
  \Phi(D_i)-\Phi(D_{i-1})&=(s+1)-s\\
  &=1
\end{align*}
Therefore the amortized cost of this \texttt{PUSH} operation is:
\begin{align*}
  \hat{c_i}&=1+1=2
\end{align*}
Let the $i$th operation on the stack be \texttt{MULTIPOP(S,k)}, and let the
potential difference be $-k'$, where $k'=\textrm{min}(k,s)$ is the amount of
objects that are popped from the stack. The amortized cost is:
\begin{align*}
  \hat{c_i}&=c_i+\Phi(D_i)-\Phi(D_{i-1})\\
  &=k'-k'\\
  &=0
\end{align*}
The amortized cost of \texttt{POP} is the same, and it is found in the same way.
\newline\newline
Since $\Phi(D_i)-\Phi(D_{i-1})$ and each operation has an amortized cost of
$O(1)$, then the total amortized cost of $n$ operation on the stack is $O(n)$,
which then is the worst-case upper bound cost.

\subsection*{Dynamic tables}
We have a table, which has a specific size. But realistically we wish to add and
delete elements from our table. But we have a table with a fixed size allocated
in our memory. If we wish to add more items, we wish to allocate a new table
with twice the size as before.
\begin{align*}
  T.num&=\textrm{ The number of elements in }T\\
  T.size&=\textrm{ The size of }T\\
  \alpha(T)&=\textrm{ The load factor: }T.num/T.size
\end{align*}
\subsubsection*{Table expansion}
If the table is empty we allocate a new table with size $1$ (denoted by
\texttt{T.size}). We then insert an object $x$ into the table. Once amount of
element in the table (denoted by \texttt{T.num}) is equal to \texttt{T.size},
then we allocate space for a new table of size $size_i=2\cdot size_{i-1}$, copy
all the elements in the former table into the new one, free the old table.
Lastly we add the new element to to the table. A table expansion happens, when
the load factor is $\alpha(T)=1$.
\subsubsection*{Table contraction}
We also wish to delete elements from our table, but we don't want a huge table
with few elements. So whenever the load is $\alpha(T)=1/4$, then a contraction
happens. This process is similiar to the expansion.

\subsubsection*{Proof for the amortized cost of $n$ \texttt{INSERT} and
\texttt{DELETE} operations}
Before a expansion and contraction the potential of $T$ will be $\Phi(T)=T.num$,
and thus the potential can pay for an expansion or contraction. The potential
depends on whether the load factor is $\alpha(T)<1/2$ or $\alpha(T)\geq1/2$.
\begin{align*}
  \Phi(T)=
  \begin{cases}
    2\cdot T.num-T.size &\textrm{ if }\alpha(T)\geq1/2\\
    T.size/2-T.num&\textrm{ if }\alpha(T)<1/2
  \end{cases}
\end{align*}
If $1/2\leq\alpha(T)<1$ then an \texttt{INSERT} operation, as the $i$th
operation with an actual cost of $O(1)$, will not trigger an expansion. We note
that without an expansion $size_{i-1}=size_i$. Furthermore $num_{i-1}=num_i-1$.
The amortized cost will be:
\begin{align*}
  \hat{c_i}&=c_i+\Phi_i-\Phi_{i-1}\\
  &=1+(2\cdot num_i-size_i)-(2\cdot num_{i-1}-size_{i-1})\\
  &=1+(2\cdot num_i-size_i)-(2(num_i-1)-size_i)\\
  &=1+2\cdot num_i-size_i-2\cdot num_i+2+size_i\\
  &=3
\end{align*}
If the $i$th operation does trigger an expansion, then $size_i=2size_{i-1}$ and
$size_{i-1}=num_{i-1}=num_i-1$, thus $size_i=2\cdot(num_i-1)$.
\begin{align*}
  \hat{c_i}&=c_i+\Phi_i-\Phi_{i-1}\\
  &=num_i+(2\cdot num_i-size_i)-(2\cdot num_{i-1}-size_{i-1})\\
  &=num_i+(2\cdot num_i-2(num_i-1))-(2\cdot num_{i-1}-(num_i-1))\\
  &=num_i+2-(num_i-1)\\
  &=3
\end{align*}
If the $i$th operation is \texttt{DELETE} and we consider whether
$\alpha(T)<1/2$ or $\alpha(T)\geq1/2$. Here $\alpha(T)<1/2$ and the operation
does not trigger a contraction. Therefore $size_i=size_{i-1}$ and $num_{i-1}=
num_i+1$. The amortized cost is:
\begin{align*}
  \hat{c_i}&=c_i+\Phi_i-\Phi_{i-1}\\
  &=1+(size_i/2-num_i)-(size_{i-1}/2-num_{i-1})\\
  &=1+(size_i/2-num_i)-(size_i/2-(num_i+1))\\
  &=2
\end{align*}
If $\alpha(T_{i-1})<1/2$ and the $i$th operation does trigger a contraction,
then the actual cost is $c_i=num_i+1$, since we move $num_i$ elements and delete
$1$. Furthermore $size_i/2=size_{i-1}/4=num_{i-1}=num_i$. The amortized cost is:
\begin{align*}
  \hat{c_i}&=c_i+\Phi_i-\Phi_{i-1}\\
  &=(num_i+1)+(size_i/2-num_i)-(size_{i-1}/2-num_{i-1})\\
  &=(num_i+1)+((num_i+1).num_i)-((2\cdot num_i+2)-(num_i+1))\\
  &=1
\end{align*}
The worst-case upper bound is $O(1)$, and therefore the total amortized cost for
$n$ operations on a dynamic table is $O(n)$.
