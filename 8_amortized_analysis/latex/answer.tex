\subsection*{Amortized Analysis - Stack operations}
\subsubsection*{Aggregate analysis}
We determine an upper bound $T(n)$ on the total cost of a sequence of $n$
operation. We take the average cost as the amortized cost of each operation, so
that all operations have the same amortized cost. Even several types of
operations will have the same amortized cost.
\newline\newline
For a sequence of $n$ \texttt{PUSH}, \texttt{POP} and \texttt{MULTIPOP}
operations on an empty stack, all operations have a total cost of $O(n)$, since
the number of calls within \texttt{MULTIPOP}, and the number of times
\texttt{POP} is called, cannot exceed the number of \texttt{PUSH operations},
which is at most $n$. The average cost of each operation is $O(n)/n=O(1)$.

\subsubsection*{Accounting method}
In the accounting method we charge early operations more than their cost. That
way we charge less in later opeartions. Each operation has its own amortized
cost, and we save the extra charge as "prepaid credit".
\newline\newline

\subsection*{Dynamic tables}
The potential method works in the same way as the accounting method, but
maintains the extra credit as "potential energy". This is associated with the
data structure as a whole, instead of individual operations.
\newline\newline
