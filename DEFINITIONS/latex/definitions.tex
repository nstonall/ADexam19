\subsection*{Runtime}
$cn=\Theta(n)$\\
$dn=O(n)$
\newline\newline
The runtime of a \textbf{Divide-and-Conquer} algorithm is expressed as such:
\begin{align*}
  T(n)=
  \begin{cases}
    \Theta(n)&\textrm{ if }n\leq c\\
    aT(n/b)+D(n)+C(n)&\textrm{ otherwise.}
  \end{cases}
\end{align*}
\subsection*{Fibonacci Heaps}
A \textit{(binary) heap} is an array structure that looks like a binary tree.
It has two atrributes; $A.length$ and $A.heap-size$, where $0\leq
A[1..A.heap-size]\leq A[1..A.length]$.\newline
Although $A[1..A.length]$ may contain elements, only the elements in
$A[1..A.heap-size]$ are valid elements of the heap.
\begin{figure}[H]
  \includegraphics{pictures/heap.png}
  \caption{Max-heap}
\end{figure}
